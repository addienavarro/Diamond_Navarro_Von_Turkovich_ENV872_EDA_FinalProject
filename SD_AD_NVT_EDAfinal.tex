% Options for packages loaded elsewhere
\PassOptionsToPackage{unicode}{hyperref}
\PassOptionsToPackage{hyphens}{url}
%
\documentclass[
]{article}
\usepackage{amsmath,amssymb}
\usepackage{lmodern}
\usepackage{ifxetex,ifluatex}
\ifnum 0\ifxetex 1\fi\ifluatex 1\fi=0 % if pdftex
  \usepackage[T1]{fontenc}
  \usepackage[utf8]{inputenc}
  \usepackage{textcomp} % provide euro and other symbols
\else % if luatex or xetex
  \usepackage{unicode-math}
  \defaultfontfeatures{Scale=MatchLowercase}
  \defaultfontfeatures[\rmfamily]{Ligatures=TeX,Scale=1}
\fi
% Use upquote if available, for straight quotes in verbatim environments
\IfFileExists{upquote.sty}{\usepackage{upquote}}{}
\IfFileExists{microtype.sty}{% use microtype if available
  \usepackage[]{microtype}
  \UseMicrotypeSet[protrusion]{basicmath} % disable protrusion for tt fonts
}{}
\makeatletter
\@ifundefined{KOMAClassName}{% if non-KOMA class
  \IfFileExists{parskip.sty}{%
    \usepackage{parskip}
  }{% else
    \setlength{\parindent}{0pt}
    \setlength{\parskip}{6pt plus 2pt minus 1pt}}
}{% if KOMA class
  \KOMAoptions{parskip=half}}
\makeatother
\usepackage{xcolor}
\IfFileExists{xurl.sty}{\usepackage{xurl}}{} % add URL line breaks if available
\IfFileExists{bookmark.sty}{\usepackage{bookmark}}{\usepackage{hyperref}}
\hypersetup{
  pdftitle={Exploring Flood Risk Ashville, NC},
  pdfauthor={Sarah Diamond, Addie Navaro, Natalie von Turkovich},
  hidelinks,
  pdfcreator={LaTeX via pandoc}}
\urlstyle{same} % disable monospaced font for URLs
\usepackage[margin=2.54cm]{geometry}
\usepackage{graphicx}
\makeatletter
\def\maxwidth{\ifdim\Gin@nat@width>\linewidth\linewidth\else\Gin@nat@width\fi}
\def\maxheight{\ifdim\Gin@nat@height>\textheight\textheight\else\Gin@nat@height\fi}
\makeatother
% Scale images if necessary, so that they will not overflow the page
% margins by default, and it is still possible to overwrite the defaults
% using explicit options in \includegraphics[width, height, ...]{}
\setkeys{Gin}{width=\maxwidth,height=\maxheight,keepaspectratio}
% Set default figure placement to htbp
\makeatletter
\def\fps@figure{htbp}
\makeatother
\setlength{\emergencystretch}{3em} % prevent overfull lines
\providecommand{\tightlist}{%
  \setlength{\itemsep}{0pt}\setlength{\parskip}{0pt}}
\setcounter{secnumdepth}{5}
\usepackage{booktabs}
\usepackage{longtable}
\usepackage{array}
\usepackage{multirow}
\usepackage{wrapfig}
\usepackage{float}
\usepackage{colortbl}
\usepackage{pdflscape}
\usepackage{tabu}
\usepackage{threeparttable}
\usepackage{threeparttablex}
\usepackage[normalem]{ulem}
\usepackage{makecell}
\usepackage{xcolor}
\ifluatex
  \usepackage{selnolig}  % disable illegal ligatures
\fi

\title{\textbf{Exploring Flood Risk Ashville, NC}}
\usepackage{etoolbox}
\makeatletter
\providecommand{\subtitle}[1]{% add subtitle to \maketitle
  \apptocmd{\@title}{\par {\large #1 \par}}{}{}
}
\makeatother
\subtitle{\url{https://github.com/addienavarro/Diamond_Navarro_Von_Turkovich_ENV872_EDA_FinalProject}}
\author{Sarah Diamond, Addie Navaro, Natalie von Turkovich}
\date{}

\begin{document}
\maketitle

\newpage
\tableofcontents 
\newpage
\listoftables 
\newpage
\listoffigures 
\newpage

\hypertarget{rationale-and-research-questions}{%
\subsection{\texorpdfstring{\textbf{Rationale and Research
Questions}}{Rationale and Research Questions}}\label{rationale-and-research-questions}}

The earths climate is changing and is resulting higher seas, new weather
patterns and stronger storms (Floodfactor, 2022). The warming atmosphere
is causeing more evaporation, which leads to more water availble for
precipitation (Floodfactor, 2022). This is resulting in more extreme
weather events. Both the frequency and magnitude of weather events is
creasing world wide. North Carolina has not be spared this new weather
patterns. Intense rains in North Carolina have been causing flooding.
Devastating flooding as occurred recently in the mountainous, western
portion of the state. Flooding in Asheville, located in the western
North Carolina along the French Broad River resulted in two fatalities
in the fall of 2022 (Harris, 2021). The the orographic rains that this
mountainous region is prone along with the topography of the terrain to
contributes to this areas vulnerability to flooding.

In light of the recent flooding in Asheville we are interested in
exploring if flood risk in Asheville, NC is increasing over time. To do
this we will analyze both precipitation data in Asheville as well as
river discharge data on the French Broad River. For data set we will
look at NOAA precipitation data, and USGS stream gage data. We will be
asking the following questions:

\begin{enumerate}
\def\labelenumi{\arabic{enumi}.}
\item
  Is discharge increasing over time?
\item
  What trends exist in the discharge data over time?
\item
  Is precipitation increasing over time?
\item
  Are the frequency of significant precipitation events increasing over
  time?
\item
  Is the magnitude of significant rainfall events increasing over time?
\item
  Does precipitation have a significant effect on rain fall?
\end{enumerate}

\begin{figure}
\includegraphics[width=1\linewidth]{Photo_2} \caption{Flooding in Asheville.}\label{fig:unnamed-chunk-2}
\end{figure}

\newpage

\hypertarget{dataset-information}{%
\subsection{\texorpdfstring{\textbf{Dataset
Information}}{Dataset Information}}\label{dataset-information}}

\hypertarget{discharge-data}{%
\subsubsection{\texorpdfstring{\textbf{Discharge
Data}}{Discharge Data}}\label{discharge-data}}

To understand the discharge in Asheville, North Carolina we used data
from the United States Geographical Survey's (USGS) National Water
Information System (NWIS). We were able to pull this dataset into R
using the dataretrieval function which lets you simply put the specific
USGS code for the area we were interested in looking more closely at,
ours being the French Broad River. We chose stream gauge station
03451500 which is close to the city center of Asheville. There were
multiple parameters available for this site including discharge,
precipitation, pH, stream level, etc. By identifying the USGS code as
well as the specific codes for the parameters we wanted to look at
(i.e., discharge data) we were able to pull in corresponding data for
the last 60 years. Because discharge data is recorded daily, we had
records of every day from 1963 to 2021. The pulled dataset included the
agency (USGS), the site number, the date, and the amount of discharge in
cubic feet per second.

Because we were able to pull in exactly which columns we wanted, there
was not much to wrangle for this specific dataset. We did however decide
that for all the parameters we were looking at that we would only
include the last forty years (1981-2021). In order to do this, we
filtered the dataset to only include those specific years. We also used
the lubridate package to change the date column to have a date class
format.

\begin{table}
\centering
\begin{tabular}[t]{l|l|l|r}
\hline
\multicolumn{4}{c}{Asheville USGS Precipitation Data} \\
\cline{1-4}
Agency Code & Site Number & Date & Discharge\\
\hline
USGS & 03451500 & 1981-01-02 & 867\\
\hline
USGS & 03451500 & 1981-01-03 & 840\\
\hline
USGS & 03451500 & 1981-01-04 & 825\\
\hline
USGS & 03451500 & 1981-01-05 & 751\\
\hline
USGS & 03451500 & 1981-01-06 & 790\\
\hline
USGS & 03451500 & 1981-01-07 & 800\\
\hline
\end{tabular}
\end{table}

\hypertarget{precipitation-data}{%
\subsubsection{\texorpdfstring{\textbf{Precipitation
Data}}{Precipitation Data}}\label{precipitation-data}}

\begin{table}
\centering
\begin{tabular}[t]{l|r|r|r}
\hline
\multicolumn{4}{c}{Asheville NOAA Precipitation Data} \\
\cline{1-4}
Date & Precip.mm & Month & Year\\
\hline
1981-01-01 & 0.3 & 1 & 1981\\
\hline
1981-01-02 & 0.0 & 1 & 1981\\
\hline
1981-01-03 & 0.0 & 1 & 1981\\
\hline
1981-01-04 & 0.0 & 1 & 1981\\
\hline
1981-01-05 & 0.0 & 1 & 1981\\
\hline
1981-01-06 & 0.0 & 1 & 1981\\
\hline
\end{tabular}
\end{table}

\hypertarget{combined-data}{%
\subsubsection{\texorpdfstring{\textbf{Combined
Data}}{Combined Data}}\label{combined-data}}

For our linear model dataset, we combined both the discharge and the
precipitation datasets into one using the leftjoin function. Doing so,
we had both the date, the amount of discharge in cubic feet per second,
and the precipitation in milliliters from the past 40 years. For some
reason, some of the precipitation entries were negative numbers so we
filtered to make sure the final dataset included only precipitation
values greater than 0 milliliters.

\begin{table}
\centering
\begin{tabular}[t]{l|r|r|r|r}
\hline
\multicolumn{5}{c}{Discharge and Precipitation Combined Data} \\
\cline{1-5}
Date & Precip.mm & Month & Year & Discharge\\
\hline
1981-01-01 & 0.3 & 1 & 1981 & NA\\
\hline
1981-01-07 & 0.3 & 1 & 1981 & 800\\
\hline
1981-01-15 & 0.5 & 1 & 1981 & 800\\
\hline
1981-01-20 & 5.6 & 1 & 1981 & 770\\
\hline
1981-01-21 & 0.3 & 1 & 1981 & 770\\
\hline
1981-01-27 & 2.3 & 1 & 1981 & 703\\
\hline
\end{tabular}
\end{table}

\newpage

\hypertarget{exploratory-analysis}{%
\subsection{\texorpdfstring{\textbf{Exploratory
Analysis}}{Exploratory Analysis}}\label{exploratory-analysis}}

\hypertarget{discharge-exploration}{%
\subsubsection{\texorpdfstring{\textbf{Discharge
Exploration}}{Discharge Exploration}}\label{discharge-exploration}}

In order to better understand the flood risk in Asheville, North
Carolina, we were interested in understanding the daily discharge data
throughout time as well as the overall relationship between discharge
and precipitation. To measure this we wanted to run a time series of the
discharge data as well as a seasonal decomposition to understand both
the seasonality of the data and the trend throughout time. We also
wanted to understand what in fact was the trend over time and how this
information can inform the flood risk for Asheville.

As for the relationship between discharge and precipitation, we felt the
best way to see this was through a general linear model. By looking at
the dependence that discharge may have over precipitation, we were
confident that this too would help us to answer our original research
question of how these two parameters and the relationship between them
have impacted the flood risk over time.

\begin{figure}
\centering
\includegraphics{SD_AD_NVT_EDAfinal_files/figure-latex/unnamed-chunk-6-1.pdf}
\caption{French Broad Discharge over time.}
\end{figure}

\hypertarget{precipitation-exploration}{%
\subsubsection{\texorpdfstring{\textbf{Precipitation
Exploration}}{Precipitation Exploration}}\label{precipitation-exploration}}

\begin{figure}
\centering
\includegraphics{SD_AD_NVT_EDAfinal_files/figure-latex/unnamed-chunk-7-1.pdf}
\caption{Asheville precipitation over time.}
\end{figure}

\newpage

\hypertarget{analysis}{%
\subsection{\texorpdfstring{\textbf{Analysis}}{Analysis}}\label{analysis}}

Through our analysis we became more familiar with the data and the
relationship between precipitation and discharge in Asheville.

The following analysis can be divided into three parts:

\begin{enumerate}
\def\labelenumi{\arabic{enumi}.}
\item
  How has discharge changed over time?
\item
  How has precipitation changed over time?
\item
  What is the relationship of discharge to precipitaion?
\end{enumerate}

\hypertarget{discharge-analysis}{%
\subsubsection{\texorpdfstring{\textbf{Discharge
Analysis}}{Discharge Analysis}}\label{discharge-analysis}}

To first look at the trends of discharge data over time, we ran a simple
timeseries that began on January 1, 1981, and ran through December 31,
2021, with a frequency of 365 days. Because this data had a seasonal
component, we examined the decomposition of this timeseries data to get
a better visual of the trend and seasonality throughout time.

\begin{figure}
\centering
\includegraphics{SD_AD_NVT_EDAfinal_files/figure-latex/unnamed-chunk-8-1.pdf}
\caption{French Broad River discharge, time series decompostion.}
\end{figure}

After this, we created an ``Observed'' column in the dataframe to show
the discharge with the corresponding date so we could visualize both the
trend and the seasonality throughout time.

\begin{figure}
\centering
\includegraphics{SD_AD_NVT_EDAfinal_files/figure-latex/unnamed-chunk-9-1.pdf}
\caption{French Broad River Discharge, trend over time with trend line.}
\end{figure}

\begin{figure}
\centering
\includegraphics{SD_AD_NVT_EDAfinal_files/figure-latex/unnamed-chunk-10-1.pdf}
\caption{French Broad River Discharge, trend over time with seasonal
variation line.}
\end{figure}

Lastly, we used the Seasonal Mann-Kendall test to know whether the trend
over time was positive or negative. Our results of this test showed that
we could reject the null hypothesis that there is no trend in the
seasonal data and that there is a positive trend over time in the
discharge data (pvalue \textgreater{} 0.05, tau = 0.0622). The final
plot of the trend can be seen below.

\includegraphics{SD_AD_NVT_EDAfinal_files/figure-latex/unnamed-chunk-11-1.pdf}
\newpage

\hypertarget{precipitation-analysis}{%
\subsubsection{\texorpdfstring{\textbf{Precipitation
Analysis}}{Precipitation Analysis}}\label{precipitation-analysis}}

\begin{table}
\centering
\begin{tabular}[t]{l|r|r|r|r|r|r|r|r|r|r}
\hline
\multicolumn{11}{c}{Significant Rainfall in Asheville, NC} \\
\cline{1-11}
Duration & 1 year & 2 year & 5 year & 10 year & 25 year & 50 year & 100 year & 200 year & 500 year & 1000 year\\
\hline
5-min: & 8 & 10 & 12 & 14 & 16 & 17 & 19 & 20 & 22 & 24\\
\hline
10-min: & 14 & 16 & 19 & 22 & 25 & 28 & 30 & 32 & 35 & 38\\
\hline
15-min: & 17 & 20 & 25 & 28 & 32 & 35 & 38 & 41 & 44 & 47\\
\hline
30-min: & 23 & 28 & 35 & 40 & 47 & 52 & 58 & 63 & 71 & 77\\
\hline
60-min: & 29 & 35 & 45 & 52 & 63 & 71 & 80 & 89 & 102 & 112\\
\hline
2-hr: & 33 & 40 & 51 & 59 & 71 & 81 & 91 & 102 & 117 & 129\\
\hline
3-hr: & 35 & 41 & 52 & 61 & 74 & 85 & 96 & 108 & 125 & 139\\
\hline
6-hr: & 41 & 48 & 60 & 70 & 84 & 96 & 109 & 123 & 143 & 159\\
\hline
12-hr: & 50 & 59 & 73 & 84 & 99 & 111 & 124 & 136 & 153 & 166\\
\hline
24-hr: & 55 & 66 & 82 & 95 & 112 & 125 & 139 & 153 & 171 & 184\\
\hline
2-day: & 66 & 79 & 97 & 111 & 130 & 145 & 160 & 175 & 194 & 209\\
\hline
3-day: & 70 & 84 & 102 & 116 & 136 & 151 & 165 & 180 & 199 & 213\\
\hline
4-day: & 74 & 89 & 107 & 122 & 141 & 156 & 171 & 185 & 204 & 217\\
\hline
7-day: & 87 & 104 & 125 & 141 & 163 & 180 & 196 & 213 & 234 & 249\\
\hline
10-day: & 100 & 118 & 141 & 158 & 181 & 199 & 216 & 234 & 255 & 271\\
\hline
20-day: & 138 & 162 & 189 & 210 & 237 & 258 & 278 & 297 & 321 & 338\\
\hline
30-day: & 172 & 201 & 230 & 252 & 279 & 299 & 318 & 335 & 357 & 371\\
\hline
45-day: & 218 & 255 & 286 & 310 & 338 & 358 & 376 & 393 & 412 & 424\\
\hline
60-day: & 262 & 305 & 341 & 366 & 398 & 420 & 439 & 457 & 477 & 490\\
\hline
\end{tabular}
\end{table}

\begin{figure}
\centering
\includegraphics{SD_AD_NVT_EDAfinal_files/figure-latex/unnamed-chunk-13-1.pdf}
\caption{Frequency and magnitude of precipitation events over time.}
\end{figure}

\newpage

\hypertarget{discharge-and-precitpitaion-relationship}{%
\subsubsection{\texorpdfstring{\textbf{Discharge and Precitpitaion
Relationship}}{Discharge and Precitpitaion Relationship}}\label{discharge-and-precitpitaion-relationship}}

In our next analysis, we looked more at the relationship between
discharge and the precipitation data. As mentioned previously, we used a
general linear model to look more closely at this relationship. Our
results of this linear model regression show that there was in fact a
significant relationship between the two variables, and we were
confidently able to reject the null hypothesis that there was no
relationship between discharge and precipitation. It can be seen below
in the QQ-Plot that this data is not really normally distributed and
that it would be advantageous to potentially log-transform the data to
view the relationship better

\begin{figure}

{\centering \includegraphics{SD_AD_NVT_EDAfinal_files/figure-latex/unnamed-chunk-14-1} 

}

\caption{French Broad River discharge and Asheville precipitation, linear model results.}\label{fig:unnamed-chunk-14}
\end{figure}

Looking closer at the summary table, we were able to determine that this
was a positive relationship and that for every one unit of
precipitation, discharge increased by 48.3 cubic feet per second
(p-value \textless{} 0.05, F-statistic = 324.3 on 1 and 525 degrees of
freedom, R\^{}2 = 0.05805). Our R-squared value however only explained
around 6\% of the variability and this could be due to the fact that
discharge and precipitation isn't usually modelled linearly which could
account for the lower R\^{}2 and could be something to investigate
further. A detailed visualization of the relationship can be seen in the
plot below.

\begin{figure}
\centering
\includegraphics{SD_AD_NVT_EDAfinal_files/figure-latex/unnamed-chunk-15-1.pdf}
\caption{French Broad River discharge by Asheville precipitation, linear
relationship.}
\end{figure}

Both of these analyses can be helpful to inform our original research
question of how these variables and the relationship between them have
changed over time and will be discussed at greater length in our summary
and conclusion sections.

\newpage

\hypertarget{summary-and-conclusions}{%
\subsection{\texorpdfstring{\textbf{Summary and
Conclusions}}{Summary and Conclusions}}\label{summary-and-conclusions}}

Summarize your major findings from your analyses in a few paragraphs.
What conclusions do you draw from your findings? Relate your findings
back to the original research questions and rationale.

Through our analysis we observed the how precipitation and river
discharge are changing in Asheville over time. Even though there may be
an increase of flooding events that does not always correlate to an
increases in precipitation and discharge as large events could be
balanced by period of lower than normal rainfall. Our analysis did allow
us to answer our research questions and gain a better understand of
hydrologic events in Asheville in the past 40 years.

Through our analysis of discharge in the French Broad river we were able
to answer our first research question of what trends exist in dishcarge
data over time. We found that athough small, there is indeed an increase
in overall discharge moving through the Asheville in the past 40
years\_\_\_. After decomposing our time series analysis we found that
the trend of the overall data was not clearly increasing. We did however
find that there was a slight visual correspondence between the seasonal
data and the overall shape of the time series plot. This finding was
confirmed with the results of the Seasonal Man Kendall test. Our results
of this test showed that we could reject the null hypothesis that there
is no trend in the seasonal data and that there is a positive trend over
time in the discharge data, answering our second research question.

Analysis of the precipitation data in Asheville resulted in a result of
a positive increase in precipitation over time, answering research
questions number three \_\_\_. Additionally, we looked at the 24 hour
rain event occurring every year, two years and five years in Asheville.
We found that over the past 40 years the size of the 24 hour rain event
was increasing for each event interval (1, 2, and 5 years). Of the
highest interval event, five year event, this size event has been
increasing in frequently over time. This analysis allowed us to answer
our fourth and firth research questions. We can conclude that the
frequency and intensity of storms is increasing over time in Asheville.

A comparisson of the relationship on precipitation on river discharge
resulted in a significant result answering our sixth research question.
As one would suspect, precipitation has a positive effect on river
discharge. This result provides further evidence that the increases
measured in precipitation event magnitude and frequency will also have
an increase in magnitude and frequency of high discharge in the river.

The changing climate puts to the test the systems that have been built
around data from the past. Acknowledging these creases in hyrodologic
activity which lead to an increase in flood risk requires municipalities
to assess their current storm water system and flood plain boundaries.
Awareness of the chanding conditions is necessary for communities to be
able to adapt and build resilience to climate risks in their areas.

\begin{figure}

{\centering \includegraphics[width=0.75\linewidth]{Photo_1} 

}

\caption{Flooding in Asheville.}\label{fig:unnamed-chunk-16}
\end{figure}

\newpage

\hypertarget{references}{%
\subsection{\texorpdfstring{\textbf{References:}}{References:}}\label{references}}

Harris, S. (2021, August 20). Asheville and Buncombe flooding related
road, park closures, Help Line. The Asheville Citizen Times. Retrieved
April 16, 2022, from
\url{https://www.citizen-times.com/story/news/2021/08/20/asheville-buncombe-nc-area-flooding-road-park-closures-help-line/8211528002/}

FloodFactor (2022). Asheville, North Carolina. Flood Factor.
\url{https://floodfactor.com/city/asheville-northcarolina/3702140_fsid}

Figure 1: Photo credit:
\url{https://www.ashevillenc.gov/department/public-works/stormwater-services-utility/flood-information/}

Figure 2: Photo credit: Amy Westmoreland

\end{document}
